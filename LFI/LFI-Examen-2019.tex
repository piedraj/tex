\documentclass[12pt]{articulo}
\usepackage{graphicx}

\baselineskip 25pt
\textwidth 15cm
\textheight 23cm
\topmargin -1.5cm
\oddsidemargin 1cm
\evensidemargin 1cm

\date{\vspace{-5ex}}

\begin{document}

\title{\bf EXAMEN Laboratorio F\'isica I\\19 de enero de 2019} 

\author{}

\maketitle

Se dispone de un generador de aire y un perfil de ala de avi\'on. La velocidad
del aire $v$ se puede obtener a partir de la medida de la diferencia de
presiones total y est\'atica,
%
\begin{eqnarray*}
\Delta P = \frac{1}{2}\rho v^2,
\end{eqnarray*}
%
siendo $\rho = 1,2~{\rm kg/m^3}$ la densidad del aire. Esta diferencia $\Delta P$
se mide usando un tubo de Pitot y un man\'ometro de tubo inclinado con una
precisi\'on de $\delta P = 2~{\rm Pa}$. La fuerza\footnote{Usar en todo el
ejercicio $g = 9,81~{\rm m/s^2}$ como valor de la aceleraci\'on de la gravedad.}
de elevaci\'on se obtiene a partir de la medida de la masa mediante una balanza
tarada con el perfil de ala, con una precisi\'on de $\delta m = 10^{-3}~{\rm kg}$.
Para un \'angulo de ataque $\alpha = 30^{\circ}$ los valores medidos son
%
\begin{eqnarray*}
\Delta P = 60 \pm 2~{\rm Pa},\\
m = -0,062 \pm 0,001~{\rm kg}.
\end{eqnarray*}

Con el tubo de Pitot y para el mismo \'angulo de ataque, se miden
(Tabla~\ref{tab:presiones}) diferencias entre la presi\'on atmosf\'erica
$P_{atm}$ y la presi\'on en los diferentes orificios de las superficies superior
e inferior del perfil de ala de avi\'on, $P_{ala}$.
\begin{table}[h!]
\begin{center}
\begin{tabular}{cc|cc}
\hline
orificios superiores & $P_{atm} - P_{ala}~/{\rm Pa}$ & orificios inferiores & $P_{atm} - P_{ala}~/{\rm Pa}$\\
\hline
1 & 44 & 1 & $-53$\\
2 & 32 & 2 & $-37$\\
3 & 19 & 3 & $-16$\\
4 &  7 & 4 &  $-4$\\
\hline
\end{tabular}
\end{center}
\caption{Diferencias de presi\'on medidas en los orificios del perfil de ala de avi\'on.}
\label{tab:presiones}
\end{table}

\begin{enumerate}
\item{Obtener la fuerza de elevaci\'on ejercida sobre el ala a partir de la
medida de la masa. Calcular el correspondiente coeficiente de elevaci\'on con su
error, tomando $S = 0,016~{\rm m^2}$ como superficie del ala.}
\item{Estimar la fuerza de elevaci\'on a partir de las diferencias de presi\'on
de la Tabla~\ref{tab:presiones}. Comparar con la fuerza de elevaci\'on obtenida
a partir de la masa.}
\end{enumerate}

\end{document}
