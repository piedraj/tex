%  sample eprint article in LaTeX           --- M. Peskin, 9/7/00
%  modified for LHCP2017, lhcp2017@sjtu.edu.cn
%  This file is part of a tar file, which can be downloaded from the LHCP2017 indico site. 
%   https://indico.cern.ch/event/517784/overview 
% 


\documentclass[10pt]{article}
\usepackage{graphicx}



%%%%%%%%%%%%%%%%%%%%%%%%%%%%%%%%%%%%%%%%%%%%%%%%%%%%%%%%%%%%%%%%%%%%%%%%%%%%
%   document style macros
%%%%%%%%%%%%%%%%%%%%%%%%%%%%%%%%%%%%%%%%%%%%%%%%%%%%%%%%%%%%%%%%%%%%%%%%%%%%
\def\Title#1{\begin{center} {\Large #1 } \end{center}}
\def\Author#1{\begin{center}{ \sc #1} \end{center}}
\def\Address#1{\begin{center}{ \it #1} \end{center}}
\def\andauth{\begin{center}{and} \end{center}}
\def\submit#1{\begin{center}Submitted to {\sl #1} \end{center}}
\newcommand\pubblock{\rightline{\begin{tabular}{l} Proceedings of the Fifth Annual LHCP\\ \pubnumber\\
         \pubdate  \end{tabular}}}

\newenvironment{Abstract}{\begin{quotation} \begin{center} 
             \large ABSTRACT \end{center}\bigskip 
      \begin{center}\begin{large}}{\end{large}\end{center} \end{quotation}}

\newenvironment{Presented}{\begin{quotation} \begin{center} 
             PRESENTED AT\end{center}\bigskip 
      \begin{center}\begin{large}}{\end{large}\end{center} \end{quotation}}

\def\Acknowledgements{\bigskip  \bigskip \begin{center} \begin{large}
             \bf ACKNOWLEDGEMENTS \end{large}\end{center}}
%%%%%%%%%%%%%%%%%%%%%%%%%%%%%%%%%%%%%%%%%%%%%%%%%%%%%%%%%%%%%%%%%%%%%%%%%%%%
%  personal abbreviations and macros
%    the following package contains macros used in this document:
\input econfmacros.tex
%%%%%%%%%%%%%%%%%%%%%%%%%%%%%%%%%%%%%%%%%%%%%%%%%%%%%%%%%%%%%%%%%%%%%%%%%%%

\textwidth=6.5in  \textheight=8.75in
\hoffset=-.85in
\voffset=-0.6in

%%  DO NOT CHANGE anything above.

% include packages you will need
\usepackage{color}


%%%%%%%%%%%%%%%%%%%%%%%%%%%%%%%%%%%%%%%%%%%%%%%%%%%%%%%%%%%%%%%%%%%%
% basic data for the eprint:
%%%%%%%%%%%%%%%%%%%%%%%%%%%%%%%%%%%%%%%%%%%%%%%%%%%%%%%%%%%%%%%%%%%%

% Instruction:
% Please change each of the following fields:
%

%% preprint number data:
% If there is a preprint number from your institute, or experiment note number, please fill it in 
\newcommand\pubnumber{CMS-CR-2017/XXX}
% \newcommand\pubnumber{ }

%% date
\newcommand\pubdate{\today}

%%  Affiliation
\def\affiliation{
On behalf of the ATLAS and CMS Collaborations, \\
Instituto de F\'isica de Cantabria \\
(CSIC - Universidad de Cantabria), Spain}

%% Acknowledge the support
%%% \def\support{\footnote{Work supported by XYZ Foundation.}}



\begin{document}

% large size for the first page
\large
\begin{titlepage}
\pubblock


%% Change the title, name, abstract
%% Title 
\vfill
\Title{New results on Higgs properties}
\vfill

%  if you need to add the support use this, fill the \support definition above. 
%   \Author{ FIRSTNAME LASTNAME \support }
\Author{J\'onatan Piedra}
\Address{\affiliation}
\vfill
\begin{Abstract}
We present the latest ATLAS and CMS measurements of several Higgs properties,
such as signal-strength modifiers for the main production modes, fiducial and
differential cross sections, and the Higgs mass. We have analyzed the 13~TeV
proton-proton LHC collision data recorded in 2016, corresponding to integrated
luminosities up to 36.1~${\rm fb}^{-1}$. Results for the
${\rm H\to ZZ}\to 4\ell$ (${\rm \ell = e\mu}$),
${\rm H}\to\gamma\gamma$, and
${\rm H}\to\tau\tau$
decay channels are presented. In addition, searches for new phenomena in the
${\rm H}\to\gamma\gamma + E_{\rm T}^{\rm miss}$ and
${\rm H}\to{\rm b\bar{b}} + E_{\rm T}^{\rm miss}$ decay channels are presented.
\end{Abstract}
\vfill

% DO NOT CHANGE 
\begin{Presented}
The Fifth Annual Conference\\
 on Large Hadron Collider Physics \\
Shanghai Jiao Tong University, Shanghai, China\\ 
May 15-20, 2017
\end{Presented}
\vfill
\end{titlepage}
\def\thefootnote{\fnsymbol{footnote}}
\setcounter{footnote}{0}
%

% normal size for the rest
\normalsize 

%% Your paper should be entered below. 


%~~~~~~~~~~~~~~~~~~~~~~~~~~~~~~~~~~~~~~~~~~~~~~~~~~~~~~~~~~~~~~~~~~~~~~~~~~~~~~~
\section{Introduction}
%~~~~~~~~~~~~~~~~~~~~~~~~~~~~~~~~~~~~~~~~~~~~~~~~~~~~~~~~~~~~~~~~~~~~~~~~~~~~~~~

The discovery of the Higgs boson was announced in 2012 by the ATLAS and CMS
collaborations~\cite{Aad:2012tfa,Chatrchyan:2012ufa} based on proton-proton
collisions collected at the CERN LHC at the centre of mass energies of 7 and
8~TeV. Since then a huge effort has been made in the determination of the
properties of this newly found particle. The dataset already collected at 13~TeV
allows inclusive Higgs boson measurements to be repeated. Furthermore, the
increased centre-of-mass energy results in much larger cross sections for events
at high partonic centre-of-mass energy. This implies improved sensitivity to a
variety of interesting physics processes, such as Higgs bosons produced at high
transverse momentum.

In this document we present the latest ATLAS and CMS measurements of several
Higgs properties in different decay channels, such as ${\rm H\to ZZ}$,
${\rm H\to\gamma\gamma}$ and ${\rm H\to\tau\tau}$. In addition, we also present
results on searches for phenomena beyond the Standard Model, in Higgs decays
to $\gamma\gamma$ or ${\rm b\bar{b}}$, with $E_{\rm T}^{\rm miss}$ in the final
state.


%~~~~~~~~~~~~~~~~~~~~~~~~~~~~~~~~~~~~~~~~~~~~~~~~~~~~~~~~~~~~~~~~~~~~~~~~~~~~~~~
\section{\boldmath ${\rm H}\to{\rm ZZ}$}
%~~~~~~~~~~~~~~~~~~~~~~~~~~~~~~~~~~~~~~~~~~~~~~~~~~~~~~~~~~~~~~~~~~~~~~~~~~~~~~~

The ${\rm H\to ZZ\to 4\ell}$ decay channel (${\rm \ell = e,\mu}$) has a large
signal-to-background ratio due to the complete reconstruction of the final state
decay products and excellent lepton momentum resolution, making it one of the
most important channels for studies of the Higgs boson's properties. Here we
present measurements of properties of the Higgs boson in this channel at 13~TeV,
for both the ATLAS and CMS collaborations~\cite{ATLAS-ZZ,CMS:2017jkd}.

See Figure~\ref{fig:figure-ZZ}. 

\begin{figure}[htb]
\centering
\includegraphics[height=2in]{figures/ATLAS-CONF-2017-032__fig_01__m4l.pdf}
\includegraphics[height=2in]{figures/ATLAS-CONF-2017-032__fig_08a__pT4l.pdf}
\includegraphics[height=2in]{figures/ATLAS-CONF-2017-032__fig_09a__njets.pdf}\\
\includegraphics[height=2in]{figures/CMS-HIG-16-041__Figure_003-b__m4l.pdf}
\includegraphics[height=2in]{figures/CMS-HIG-16-041__Figure_009-b__pT4l.pdf}
\includegraphics[height=2in]{figures/CMS-HIG-16-041__Figure_009-c__njets.pdf}
\caption{
  (Top left) ATLAS four-lepton invariant mass distribution of the selected
  events. The systematic uncertainty on the prediction is shown by the dashed
  band.
  (Top center and right) ATLAS differential fiducial cross sections, for the
  transverse momentum of the Higgs boson (center) and the number of jets (right).
  The measured cross sections are compared to different ggH predictions, and
  predictions for all other Higgs production modes XH are added.
  (Bottom left) CMS four-lepton invariant mass distribution of the selected
  events.
  (Bottom center and right) CMS differential fiducial cross sections, for the
  transverse momentum of the Higgs boson (center) and the number of jets (right).
  The sub-dominant component of the signal (VBF + VH + ttH) is denoted as XH.
}
\label{fig:figure-ZZ}
\end{figure}


%~~~~~~~~~~~~~~~~~~~~~~~~~~~~~~~~~~~~~~~~~~~~~~~~~~~~~~~~~~~~~~~~~~~~~~~~~~~~~~~
\section{\boldmath ${\rm H}\to\gamma\gamma$}
%~~~~~~~~~~~~~~~~~~~~~~~~~~~~~~~~~~~~~~~~~~~~~~~~~~~~~~~~~~~~~~~~~~~~~~~~~~~~~~~

See Figure~\ref{fig:figure-gg}.

\begin{figure}[htb]
\centering
\includegraphics[height=2in]{figures/ATLAS-CONF-2016-067__fig_07__mgg.pdf}
\includegraphics[height=2in]{figures/ATLAS-CONF-2016-067__fig_10a__pTgg.pdf}
\includegraphics[height=2in]{figures/ATLAS-CONF-2016-067__fig_11b__njets.pdf}\\
\includegraphics[height=2in]{figures/CMS-HIG-16-040__Figure_013-b__mgg.pdf}
\includegraphics[height=2in]{figures/CMS-HIG-17-015__Figure_004-a__pTgg.pdf}
\includegraphics[height=2in]{figures/CMS-HIG-17-015__Figure_004-b__njets.pdf}
\caption{
  (Top left) ATLAS diphoton invariant mass spectrum.
  (Top center and right) ATLAS differential fiducial cross sections, for the
  transverse momentum of the Higgs boson (center) and the number of jets (right).
  (Bottom left) CMS diphoton invariant mass spectrum.
  (Bottom center and right) CMS differential fiducial cross sections, for the
  transverse momentum of the Higgs boson (center) and the number of jets (right).
}
\label{fig:figure-gg}
\end{figure}


%~~~~~~~~~~~~~~~~~~~~~~~~~~~~~~~~~~~~~~~~~~~~~~~~~~~~~~~~~~~~~~~~~~~~~~~~~~~~~~~
\section{\boldmath ${\rm H}\to\tau\tau$}
%~~~~~~~~~~~~~~~~~~~~~~~~~~~~~~~~~~~~~~~~~~~~~~~~~~~~~~~~~~~~~~~~~~~~~~~~~~~~~~~

To establish the mass generation mechanism for fermions, it is necessary to
demonstrate the direct coupling of the scalar boson to fermions, and the
proportionality of its strength to the fermion mass. The most promising decay
channel is $\tau\tau$, because of the large event rate expected in the SM
compared to the other leptonic decay modes, and of the smaller contribution
from background events with respect to the ${\rm b\bar{b}}$ channel. Here we
report the results of a search for the SM scalar boson using $35.9~{\rm fb^{-1}}$
at 13 TeV, when it decays to a pair of $\tau$ leptons~\cite{CMS:2017wyg}. The
four $\tau$-pair final states with the largest branching fractions,
$\mu\tau_{\rm h}$, ${\rm e}\tau_{\rm h}$, $\tau_{\rm h}\tau_{\rm h}$, and
${\rm e}\mu$, are studied.

%%%
%%% TO BE FILLED
%%%

The search for an excess of SM scalar boson events over the expected background
involves a global maximum likelihood fit based on two-dimensional distributions
in all channels, together with control regions for the ${\rm t{\bar t}}$, QCD
multijet and W+jets backgrounds. Figure~\ref{fig:tauhtauh_VBF} shows the
distribution observed, together with the expected background and signal
distributions, in the $\tau_{\rm h}\tau_{\rm h}$ channel and VBF category. The
signal prediction for a scalar boson with $m_{\rm H} = 125$~GeV is
normalized to its best-fit cross section times branching fraction. The background
distributions are adjusted to the results of the global maximum likelihood fit.

\begin{figure}[htb]
\centering
\includegraphics[height=2in]{figures/CMS-HIG-16-043__Figure_013__tauhtauh-VBF.pdf}
\caption{Observed and predicted 2D distributions in the VBF category of the
$\tau_{\rm h}\tau_{\rm h}$ final state. The normalization of the predicted
background distributions corresponds to the result of the global fit. The signal
distribution is normalized to its best-fit signal strength.
}
\label{fig:tauhtauh_VBF}
\end{figure}


%~~~~~~~~~~~~~~~~~~~~~~~~~~~~~~~~~~~~~~~~~~~~~~~~~~~~~~~~~~~~~~~~~~~~~~~~~~~~~~~
\section{Searches for new phenomena}
%~~~~~~~~~~~~~~~~~~~~~~~~~~~~~~~~~~~~~~~~~~~~~~~~~~~~~~~~~~~~~~~~~~~~~~~~~~~~~~~

Many searches for dark matter (DM) at the LHC involve missing transverse momentum
produced in association to detectable particles. Here we present an updated search
by the ATLAS experiment~\cite{ATLAS-twophoton-BSM} for DM associated with the SM
Higgs boson decaying to a pair of photons using $36.1~{\rm fb^{-1}}$ of pp
collision data collected at 13~TeV. Three theoretical benchmark models are
considered in this analysis. In the ${\rm Z_{B}'}$ model a massive vector
mediator ${\rm Z'}$ emits a Higgs boson and subsequently decays to a pair of Dirac
fermionic DM candidates.
%In scenarios where the DM couples to the SM only via the ${\rm Z'}$ boson, the
%associated $U'(1)$ symmetry ensures the stability of the DM particle.
The ${\rm Z'}$-${\rm 2HDM}$ model involves the ${\rm Z'}$ boson decaying
to the Higgs boson and an intermediate heavy pseudoscalar boson ${\rm A^{0}}$,
which then decays to a pair of Dirac fermionic DM particles. The third model,
referred to as the heavy scalar model, introduces a heavy scalar boson H
produced primarily via gluon-gluon fusion. Here an effective quartic coupling
between the SM Higgs h, H and two DM particles is considered, with the DM
particle assumed to be scalar. The events that pass a common selection requiring
at least two photon candidates are divided into five categories based on the
event kinematics. These categories have been optimized based either on the ${\rm Z_{B}'}$
and ${\rm Z'}$-${\rm 2HDM}$ signal samples, or using simulated heavy scalar
boson samples, to cover the different kinematic regimes of the heavy scalar
model. The results of the analysis are derived from a likelihood fit of the
$m_{\gamma\gamma}$ distribution in the range of
$105~{\rm GeV} < m_{\gamma\gamma} < 160~{\rm GeV}$.
%The systematic uncertainties worsen the limits by about 10\% (7\% from the
%non-resonant background modelling and 3\% from the other systematic uncertainties).
No significant excess over
the background expectation is observed and 95\% confidence level limits are set
on the production cross section times branching fraction of the SM Higgs boson
decaying into two photons in association with missing transverse energy in the
three different theoretical benchmark models. 95\% confidence level limits are
also set on the observed signal strength in a two-dimensional
$m_{\chi}$-$m_{\rm Z_{B}'}$ plane for the ${\rm Z_{B}'}$ model, and the
$m_{\rm A^0}$-$m_{\rm Z'}$ plane for the ${\rm Z'}$-${\rm 2HDM}$ model. In the
model involving heavy scalar production, 95\% confidence level upper limits are
set on the production cross section times the branching fraction of
${\rm H}\to{\rm h}\chi\chi$, for a dark matter
particle with mass of 60 GeV. The heavy scalar model is excluded for all the
benchmark points investigated.

Another search for DM has been carried out in~\cite{ATLAS-bb-BSM}. In this case
DM is searched for in association with a SM-like Higgs boson decaying to a pair
of b-quarks, using $36.1~{\rm fb^{-1}}$ of pp collisions at 13 TeV recorded with
the ATLAS detector. The ${\rm Z'}$-${\rm 2HDM}$ model has been used for the
optimization of the search and its interpretation. Multivariate algorithms are
used to identify jets containing b-hadrons that are expected in
${\rm h}\to{\rm b{\bar b}}$ decays.


%~~~~~~~~~~~~~~~~~~~~~~~~~~~~~~~~~~~~~~~~~~~~~~~~~~~~~~~~~~~~~~~~~~~~~~~~~~~~~~~
\section{Conclusions}
%~~~~~~~~~~~~~~~~~~~~~~~~~~~~~~~~~~~~~~~~~~~~~~~~~~~~~~~~~~~~~~~~~~~~~~~~~~~~~~~


%%  if necessary
%% J.Piedra %% \Acknowledgements
%% J.Piedra %% I am grateful to XYZ for fruitful discussions.


\begin{thebibliography}{99}

%%
%%  bibliographic items can be constructed using the LaTeX format in SPIRES:
%%    see    http://www.slac.stanford.edu/spires/hep/latex.html
%%  SPIRES will also supply the CITATION line information; please include it.
%%

\bibitem{Aad:2012tfa} 
  G.~Aad {\it et al.}  [ATLAS Collaboration],
  %``Observation of a new particle in the search for the Standard Model Higgs boson with the ATLAS detector at the LHC,''
  Phys.\ Lett.\ B {\bf 716}, 1 (2012)
  [arXiv:1207.7214 [hep-ex]].
  %%CITATION = ARXIV:1207.7214;%%
  %3009 citations counted in INSPIRE as of 22 Jul 2014
  
  
%\cite{Chatrchyan:2012ufa}
\bibitem{Chatrchyan:2012ufa} 
  S.~Chatrchyan {\it et al.}  [CMS Collaboration],
  %``Observation of a new boson at a mass of 125 GeV with the CMS experiment at the LHC,''
  Phys.\ Lett.\ B {\bf 716}, 30 (2012)
  [arXiv:1207.7235 [hep-ex]].
  %%CITATION = ARXIV:1207.7235;%%
  %2951 citations counted in INSPIRE as of 22 Jul 2014


\bibitem{ATLAS-ZZ}
  G.~Aad {\it et al.}  [ATLAS Collaboration],
  ATLAS-HIGG-2016-25.


%\cite{CMS:2017jkd}
\bibitem{CMS:2017jkd}
  CMS Collaboration [CMS Collaboration],
  %``Measurements of properties of the Higgs boson decaying into four leptons in pp collisions at sqrt{s} = 13 TeV,''
  CMS-PAS-HIG-16-041.
  %%CITATION = CMS-PAS-HIG-16-041;%%
  %4 citations counted in INSPIRE as of 29 May 2017


%\cite{CMS:2017wyg}
\bibitem{CMS:2017wyg}
  CMS Collaboration [CMS Collaboration],
  %``Observation of the SM scalar boson decaying to a pair of $\tau$ leptons with the CMS experiment at the LHC,''
  CMS-PAS-HIG-16-043.
  %%CITATION = CMS-PAS-HIG-16-043;%%


\bibitem{ATLAS-twophoton-BSM}
  G.~Aad {\it et al.}  [ATLAS Collaboration],
  ATLAS-CONF-2017-024.


\bibitem{ATLAS-bb-BSM}
  G.~Aad {\it et al.}  [ATLAS Collaboration],
  ATLAS-CONF-2017-028.


\end{thebibliography}

 
\end{document}
