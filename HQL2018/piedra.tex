% Please make sure you insert your
% data according to the instructions in PoSauthmanual.pdf
\documentclass{PoS}

\title{Rare decays at CMS}

\ShortTitle{Rare decays at CMS}

\author{\speaker{J\'onatan Piedra}\thanks{On behalf of the CMS Collaboration.}\\
        IFCA (CSIC - Universidad de Cantabria)\\
        E-mail: \email{piedra@cern.ch}}

%\author{Another Author\\
%        Affiliation\\
%        E-mail: \email{...}}

\abstract{CMS searches for flavour changing neutral currents (FCNC) in events\
with top quarks and the Z or the Higgs boson are presented. Upper limits\
at 95\% confidence level are set on the branching fractions of\
${\rm t \to qZ}$ and ${\rm t \to qH}$ decays. In addition, angular distributions\
of ${\rm B^0}$ and ${\rm B^+}$ decays are presented. These measurements are found\
to be consistent with predictions based on the standard model.}

\FullConference{XIV International Conference on Heavy Quarks and Leptons (HQL2018)\\
		May 27 - June 1, 2018\\
		Yamagata Terrsa, Yamagata, Japan}


\begin{document}


%-------------------------------------------------------------------------------
\section{Introduction}
%-------------------------------------------------------------------------------

In the standard model (SM), flavour changing neutral currents (FCNC) are forbidden
at tree level and highly suppressed at higher order. Several extensions of the SM
enhance the FCNC branching fractions and can be probed at the LHC; the new couplings
can also provide for flavour changing single top quark production in association
with a Z or a Higgs boson.


%-------------------------------------------------------------------------------
\section{FCNC in ${\rm tZq \to 3\ell}$}
%-------------------------------------------------------------------------------

This analysis~\cite{top-17-017} uses proton-proton collision data coming from the
2016 data taking period, corresponding to an integrated luminosity of
${\rm 35.9~fb^{-1}}$ at a centre-of-mass energy of 13~TeV collected by the CMS
detector. The analysis focuses on the experimental search for evidence of a FCNC
vertex (referred to as tZq) with a top quark, a Z boson, and a quark q that is
either up or charm. In the final state we expect one jet originating from a b quark,
a W boson, a Z boson, and (in the case of FCNC top quark decay) a jet originating
from the up or charm quark. As we limit the analysis to leptonic decays of both the
W and the Z boson, events are selected requiring exactly 3 leptons containing one
opposite sign, same flavour pair; at least 1 jet and at most 3 jets; and the
transverse mass of the W boson below 300~GeV. Four different lepton channels are
considered (3e, 2e1$\mu$, 1e2$\mu$, 3$\mu$) and two signal regions, the single top
quark FCNC (STSR) and the top quark pair FCNC (TTSR), are constructed, using the
jet multiplicity. A simultaneous global fit is performed taking into account
both signal regions and background regions, for the four lepton channels with the
help of boosted decision trees (BDT), used to discriminate between signals and
backgrounds. The resulting discriminating output variables from each BDT are
shown in Figure~\ref{fig:TOP-17-017_Figure_003}. The signal strength and significance
are computed treating all systematic uncertainties as nuisance parameters and are
constrained from the fit, which uses templates in the different signal and
background regions for each of the four different lepton channels. The resulting
observed (expected) limits where both couplings are non-vanishing are shown in
Figure~\ref{fig:TOP-17-017_Figure_007}. No significant deviation is observed from
the predicted background. Observed (expected) upper limits at 95\% CL are set on
the branching fractions of top quark decays:
${\cal B}({\rm t \to uZ}) < 0.024\%~(0.015\%)$ and
${\cal B}({\rm t \to cZ}) < 0.045\%~(0.037\%)$, assuming one non-vanishing
coupling at a time.


\begin{figure}[htb]
\centering
\includegraphics[width=0.45\textwidth]{figures/CMS-PAS-TOP-17-017_Figure_003-a}
\includegraphics[width=0.45\textwidth]{figures/CMS-PAS-TOP-17-017_Figure_003-b}\\
\includegraphics[width=0.45\textwidth]{figures/CMS-PAS-TOP-17-017_Figure_003-c}
\includegraphics[width=0.45\textwidth]{figures/CMS-PAS-TOP-17-017_Figure_003-d}
\caption{
  The discriminating variable distribution after the fit for all different
  leptonic channels. Upper left: top quark pair tZu; upper right: top quark pair
  tZc; lower left: single top quark tZu; lower right: single top quark tZc.
}
\label{fig:TOP-17-017_Figure_003}
\end{figure}


\begin{figure}[htb]
\centering
\includegraphics[width=0.45\textwidth]{figures/CMS-PAS-TOP-17-017_Figure_007-a}
\includegraphics[width=0.45\textwidth]{figures/CMS-PAS-TOP-17-017_Figure_007-b}
\caption{
  Exclusion regions at 95\% CL on the FCNC branching fractions (left) and
  couplings (right) in the 2D plane of both the tZu and tZc variables. The CMS
  8 TeV observed (expected) limit is given with a blue line (dashed line).
}
\label{fig:TOP-17-017_Figure_007}
\end{figure}


%-------------------------------------------------------------------------------
\section{FCNC in ${\rm tH \to bb}$}
%-------------------------------------------------------------------------------

In this analysis~\cite{top-17-003} we also search for FCNC in events with the
top quark and the Higgs boson, but considering the Higgs boson decays to b
quarks. The tH FCNC interaction is studied in two channels: the associated
production of a single top quark with the Higgs boson (ST), and the FCNC decays
of top quarks in ${\rm t{\bar t}}$ semileptonic events (TT). As before, the data
sample corresponds to an integrated luminosity of~${\rm 35.9~fb^{-1}}$. Events
with exactly one isolated lepton (electron or muon) are selected, and at least
three jets are required to be present. As signal events contain three b quarks,
we require that at least two jets are identified as b quark jets by the combined
secondary vertex (CSV) b-tagging algorithm. In order to optimize sensitivity to
the signal event selection, events are split into five categories based on the
total number of reconstructed jets and on the number of b-tagged jets. With the
use of the energy and momenta of all particles, a full kinematic reconstruction
of the event is performed for several signal (ST and TT) and background
(${\rm t{\bar t}}$) hypotheses. The reconstruction is performed for all possible
permutations of the b-tagged jets to be associated with the decay products of
the Higgs boson or the top quark. The reconstructed kinematic variables for each
permutation are then fed  into a multivariate analysis that uses a BDT, trained
to distinguish the correct from the wrong b jet assignments. Some of the most
discriminating BDT input variables can be seen in Figure~\ref{fig:TOP-17-003_Figure_002}.

\ref{fig:TOP-17-003_Figures_006-007}
\ref{fig:fcnc_summarybsm_may18}


\begin{figure}[htb]
\centering
\includegraphics[width=0.45\textwidth]{figures/CMS-TOP-17-003_Figure_002-b}
\includegraphics[width=0.45\textwidth]{figures/CMS-TOP-17-003_Figure_002-c}
\caption{
  Comparison between data and simulation for some of the most discriminating BDT
  input variables in the category with three jets, all of them b-tagged: CSV
  discriminant value for one of the reconstructed b jets assigned to Higgs boson
  decay (left), and reconstructed invariant mass of two b jets associated with
  the Higgs boson decay (right).
}
\label{fig:TOP-17-003_Figure_002}
\end{figure}


\begin{figure}[htb]
\centering
\includegraphics[width=0.3\textwidth]{figures/CMS-TOP-17-003_Figure_006}
\includegraphics[width=0.3\textwidth]{figures/CMS-TOP-17-003_Figure_007-a}
\includegraphics[width=0.3\textwidth]{figures/CMS-TOP-17-003_Figure_007-b}
\caption{
  Upper limits on ${\cal B}({\rm t \to uH})$ and ${\cal B}({\rm t \to cH})$ at
  95 \% CL (left), and the best fit signal strength for Hut (center) and Hct
  (right), which is restricted to positive values in the fit.
}
\label{fig:TOP-17-003_Figures_006-007}
\end{figure}


\begin{figure}[htb]
\centering
\includegraphics[width=0.45\textwidth]{figures/fcnc_summarybsm_may18}
\caption{
  Summary of the current 95\% confidence level observed limits on the branching
  ratios of the top quark decays via flavour changing neutral currents to a
  quark and a neutral boson by the ATLAS and CMS Collaborations compared to
  several new physics models. 
}
\label{fig:fcnc_summarybsm_may18}
\end{figure}


%-------------------------------------------------------------------------------
\section{Angular observables in ${\rm B^+ \to K^+\mu^+\mu^-}$}
%-------------------------------------------------------------------------------

\cite{bph-15-001}
\ref{fig:BPH-15-001_Figures_003-004}
\ref{fig:BPH-15-001_Figure_005__BPH-15-008_Figure003} (top distributions).


\begin{figure}[htb]
\centering
\includegraphics[width=0.45\textwidth]{figures/CMS-BPH-15-001_Figure_003-h}
\includegraphics[width=0.45\textwidth]{figures/CMS-BPH-15-001_Figure_004-h}
\caption{
  Projections of the ${\rm K^+\mu^+\mu^-}$ invariant mass distribution (left) and the
  $\cos\theta_\ell$ distribution (right) from the two-dimensional fit of data, in
  the ${\rm 1 < q^2 < 6~GeV^2}$ range. The solid lines show the total fit, the
  shaded area the signal contribution, and the dash-dotted lines the background.
}
\label{fig:BPH-15-001_Figures_003-004}
\end{figure}




%-------------------------------------------------------------------------------
\section{Angular observables in ${\rm B^0 \to K^{*0}\mu^+\mu^-}$}
%-------------------------------------------------------------------------------

\cite{bph-15-008}

\ref{fig:BPH-15-001_Figure_005__BPH-15-008_Figure003} (bottom distributions).


\begin{figure}[htb]
\centering
\includegraphics[width=0.45\textwidth]{figures/CMS-BPH-15-001_Figure_005-a}
\includegraphics[width=0.45\textwidth]{figures/CMS-BPH-15-001_Figure_005-b}\\
\includegraphics[width=0.45\textwidth]{figures/CMS-BPH-15-008_Figure_003-a}
\includegraphics[width=0.45\textwidth]{figures/CMS-BPH-15-008_Figure_003-b}
\caption{
  Measurements of the $A_{\rm FB}$ (top left) and $F_{\rm H}$ (top right)
  parameters versus $q^2$ for ${\rm B^+ \to K^+\mu^+\mu^-}$ decays. Measurements of
  the $P_1$ (bottom left) and $P'_5$ (bottom right) angular parameters versus
  $q^2$ for ${\rm B^0 \to K^{*0}\mu^+\mu^-}$ decays. The CMS results are compared to
  SM  DHMV theoretical predictions~\cite{Descotes-Genon:2014uoa,Descotes-Genon:2015uva}
  and, for the $P_1$ and $P'_5$ parameters, they are also compared to results
  from the LHCb~\cite{LHCb} and Belle~\cite{Belle} Collaborations.
}
\label{fig:BPH-15-001_Figure_005__BPH-15-008_Figure003}
\end{figure}


%-------------------------------------------------------------------------------
\section{Conclusions}
%-------------------------------------------------------------------------------

To be filled.


%-------------------------------------------------------------------------------
\begin{thebibliography}{99}
%-------------------------------------------------------------------------------

\bibitem{top-17-017}
  CMS Collaboration,
  \emph{Search for flavour changing neutral currents in top quark production and decays with three-lepton final state using the data collected at $\sqrt{s} = 13~{\rm TeV}$},
  https://cds.cern.ch/record/2292045, CMS-PAS-TOP-17-017.
\bibitem{top-17-003}
  CMS Collaboration,
  \emph{Search for the flavor-changing neutral current interaction of the top quark and the Higgs boson which decays into a pair of b quarks at $\sqrt{s} = 13~{\rm TeV}$},
  \emph{accepted for publication in JHEP}
  https://cds.cern.ch/record/2296416, CERN-EP-2017-309
  [{\tt hep-ex/1712.02399}].
\bibitem{bph-15-001}
  CMS Collaboration,
  \emph{Angular analysis of the decay ${\rm B^+ \to K^+\mu^+\mu^-}$ at ${\rm \sqrt{s} = 8~TeV}$},
  https://cds.cern.ch/record/2621370 CERN-EP-2018-125
  [{\tt hep-ex/1806.00636}].
\bibitem{Descotes-Genon:2014uoa}
  To be filled.
\bibitem{Descotes-Genon:2015uva}
  To be filled.
\bibitem{bph-15-008}
  CMS Collaboration,
  \emph{Measurement of angular parameters from the decay ${\rm B^0 \to K^{*0}\mu^+\mu^-}$ at ${\rm \sqrt{s} = 8~TeV}$},
  https://cds.cern.ch/record/2287571, CERN-EP-2017-240
  [{\tt hep-ex/1710.02846}].  
\bibitem{LHCb}
  To be filled.
\bibitem{Belle}
  To be filled.
\end{thebibliography}

\end{document}
(https://pos.sissa.it/cgi-bin/reader/contribution.cgi?id=PoS(MC2000)002).
